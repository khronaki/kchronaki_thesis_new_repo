
To evaluate our contributions on larger systems we make use of the TaskSim simulator~\cite{AbstrLevels_TACO12,MUSA}. 
TaskSim is a trace driven simulator, that supports the specification of homogeneous or heterogeneous systems with many cores. 
The tracing overhead of the simulator is less than 10\% and the simulation is accurate as long as there is no contention in the shared memory resources on a real system~\cite{MUSA}.
By default, TaskSim allows the specification of the amount of cores and supports up to two core types in the case of heterogeneous asymmetric systems. 
This is done by specifying the number of cores of each type and their difference in performance between the different types (performance ratio) in the TaskSim configuration file.

Our evaluation consists of experiments on both symmetric and asymmetric platforms with the number of cores varying from 8 to 512.
In the case of asymmetric systems, we simulate the behaviour of an Arm big.LITTLE architecture~\cite{ARM}.


%To set the correct performance ratio between big and little cores, we measure the sequential execution time of each application on a real ARM big.LITTLE platform when running on a little and on a big core. 
%We use the Hardkernel Odroid~XU3 board that includes a Samsung Exynos 5422 chip with ARM big.LITTLE.
%The big cores run at 1.6GHz and the little cores at 800MHz.
%We compare its performance when they run on a little and on a big core.

%% This is TaskGenX specific!!
%Table~\ref{tab.apps} shows the measured performance ratio for each case.
%The average performance ratio among our 13 workloads is 3.8.
%Thus in the specification of the asymmetric systems we use as performance ratio the value 4.

To simulate our approaches using TaskSim we first run each application/input in the TaskSim trace generation mode.
This mode enables the online tracking of task duration and synchronization overheads and stores them in a trace file. 
To perform the simulation, TaskSim uses the information stored in the trace file and executes the application by providing this information to the runtime system.
For our experiments we generate three trace files for each application/input combination on a Genuine Intel 16-core machine running at 2.60GHz.