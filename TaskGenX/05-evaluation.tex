\subsection{Methodology}
\label{sec.taskgenx.methodology}
\begin{table*}[t]
	\scriptsize
	\begin{center}
		\caption{Evaluated benchmarks and relevant characteristics}
		\label{tab.apps}
		\resizebox{\textwidth}{!}{%
			\begin{tabular}{|c|c|c|c|c|c|c|c|c|}
				\hline
				\multirow{3}{*}{\parbox{15mm}{\centering Application}} & 
				\multirow{3}{*}{Problem size} & 
				\multirow{3}{*}{\parbox{10mm}{\centering \#Tasks}} & 
				\multirow{3}{*}{\parbox{17mm}{\centering Avg task CPU cycles (thousands)}} & 
				\multicolumn{3}{|c|}{\parbox{22mm}{\centering Per task overheads (CPU cycles)}} & & \\
				\cline{5-7}
				& & & & \multirow{2}{*}{\parbox{10mm}{\centering Create}} & \multirow{2}{*}{\parbox{9mm}{\centering All}} & \multirow{2}{*}{\parbox{11mm}{\centering Deps + Sched}} & \multirow{2}{*}{\parbox{15mm}{\centering Measured perf. ratio}} &
				\multirow{2}{*}{\parbox{10mm}{\centering $r(512)$}} \\
				& & & & & & & &  \\ %\hhline{~~~~~~}
				\hline
				
				\multirow{2}{*}{\parbox{20mm}{\centering Cholesky factorization}} & 32K 256 & 357\,762  & 753 & 15221 &  73286 &  58065 &  \multirow{2}{*}{\parbox{9mm}{\centering 3.5}} & 10.34 \\                                              
				& 32K 128 & 2829058 & 110 & 17992 &  58820 &  40828 & & 83.74 \\
				%& 32$\times$32 blocks of 512$\times$512 floats & 5984 & & 1\,551\,322 & 104.76 &  238.02 &  194.28  & \\ 
				\hline{}
				\multirow{2}{*}{\parbox{18mm}{\centering QR factorization}} & 16K 512 & 11\,442 & 518\,570  & 17595 & 63008 &   45413 & \multirow{2}{*}{\parbox{9mm}{\centering 6.8}} & 0.01 \\
				&  16K 128 & 707\,265 & 3\,558 & 21642 & 60777 & 39135 & & 3.11 \\
				\hline
				Blackscholes & native & 488\,202 & 348  &   29141  &  85438 &  56297 & 2.3 & 42.87   \\
				\hline
				Bodytrack & native & 329\,123 & 383 &  9\,505 &  18979 & 9474 & 4.2 & 12.70   \\ 
				%Heat diffusion & Heat &  &  &  &  &  & \\ 
				\hline
				Canneal & native & 3\,072\,002 & 67 & 25781 & 50094 &  24313 & 2.0 & 197.01   \\
				\hline
				Dedup & native & 20\,248 & 1\,532 & 1294 & 9647 &  8353 & 2.7 & 0.43   \\
				\hline 
				Ferret & native$\times$2 & 84\,002 & 29\,088 & 38913 & 98457 &  59544 & 3.6 & 0.68   \\
				\hline
				Fluidanimate & native & 128\,502 & 16\,734 & 30210 & 94079 &  64079 & 3.3 & 0.91   \\
				\hline
				Streamcluster & native & 3\,184\,654 & 161 & 6892 & 13693 &  6801 & 3.5 & 21.91   \\
				\hline
		\end{tabular}}
	\end{center}
	\vspace{-0.4cm}
\end{table*}

\subsubsection{Applications}
%\begin{itemize}
%\item Blackscholes
%\item Cholesky
%\item Canneal
%\item Fluidanimate
%\item QR Factorization
%\item Bodytrack
%\item Streamcluster
%\end{itemize}
Table~\ref{tab.apps} shows the evaluated applications, the input sizes used, and their characteristics. 
All applications are implemented using the OmpSs programming model~\cite{OpenMP4.0:Manual2015}. 
We obtain Cholesky and QR from the BAR repository~\cite{BAR} and we use the implementations of the rest of the benchmarks from the PARSECSs suite~\cite{Chasapis:TACO2016}.
More information about these applications can be found in~\cite{Chasapis:TACO2016} and~\cite{Chronaki:ICS2015}.
As the number of cores in SoCs is increasing, so does the need of available task parallelism~\cite{Sanchez:2010}. 
We choose the input sizes of the applications so that they are realistic cases for real life HPC applications and at the same time, create enough fine-grained tasks to feed up to 512 cores.
The number of tasks per application and input as well as the average per-task CPU cycles can be found on Table~\ref{tab.apps}.





\subsubsection{Simulation}
\label{sec:experimental:simulation}
To evaluate {\proposal} we make use of the trace-driven TaskSim simulator~\cite{AbstrLevels_TACO12,MUSA} which is described in section~\ref{sec.background.simulation}. 
%Added in the background section:
%TaskSim is a trace driven simulator, that supports the specification of homogeneous or heterogeneous systems with many cores. 
%The tracing overhead of the simulator is less than 10\% and the simulation is accurate as long as there is no contention in the shared memory resources on a real system~\cite{MUSA}.
%By default, TaskSim allows the specification of the amount of cores and supports up to two core types in the case of heterogeneous asymmetric systems. 
%This is done by specifying the number of cores of each type and their difference in performance between the different types (performance ratio) in the TaskSim configuration file.

We evaluate the effectiveness of TaskGenX on both symmetric and asymmetric platforms with the number of cores varying from 8 to 512.
In the case of asymmetric systems, we simulate the behavior of an Arm big.LITTLE architecture~\cite{ARM}.
To set the correct performance ratio between big and little cores, we measure the sequential execution time of each application on a real Arm big.LITTLE platform when running on a little and on a big core. 
We use the Hardkernel Odroid~XU3 board that includes a Samsung Exynos 5422 chip with Arm big.LITTLE.
The big cores run at 1.6GHz and the little cores at 800MHz.
%We compare its performance when they run on a little and on a big core.
Table~\ref{tab.apps} shows the measured performance ratio for each case.
The average performance ratio among our 11 workloads is 3.8.
Thus in the specification of the asymmetric systems we use as performance ratio the value 4.

%Added in the background:
%To simulate our approaches using TaskSim we first run each application/input in the TaskSim trace generation mode.
%This mode enables the online tracking of task duration and synchronization overheads and stores them in a trace file. 
%To perform the simulation, TaskSim uses the information stored in the trace file and executes the application by providing this information to the runtime system.
%For our experiments we generate three trace files for each application/input combination on a Genuine Intel 16-core machine running at 2.60GHz.

For the needs of this hardware-software co-design, we modify TaskSim so that it features one extra hardware accelerator (per multi-core) responsible for the fast task creation (the RTopt).
Apart from the task duration time, our modified simulator tracks the duration of the runtime overheads.
These overheads include: (a) task creation, (b) dependencies resolution, and (c) scheduling.
The RTopt core is optimized to execute task creation faster than the general purpose cores; 
to determine how much faster a task creation job is executed we use the analysis performed in Section~\ref{sec:hw_req}.

Using Equation~\ref{eq.create}, we compute the $C_{opt}(x)$ for each application according to their average task CPU cycles from Table~\ref{tab.apps} for $x=512$ cores.
$C_{gp}$ is the cost of task creation when it is performed on a general purpose core, namely the \textit{Create} column shown on Table~\ref{tab.apps}.
To have optimal results for each application on systems up to 512 cores, $C_{gp}$ needs to be reduced to $C_{opt}(512)$.
Thus the specialized hardware accelerator needs to perform task creation with a ratio $r(512) = C_{gp}/C_{opt}(512) \times$ faster than a general purpose core.

We compute $r(512)$ for each application shown on Table~\ref{tab.apps}. We observe that for the applications with a large number of per-task CPU cycles and relatively small \textit{Create} cycles (QR512, Dedup, Ferret, Fluidanimate), $r(512)$ is very close to zero, meaning that the task creation cost ($C_{gp}$) is already small enough for optimal task creation without the need of a faster hardware accelerator.
For the rest of the applications, RTopt needs to be more efficient. %powerful hardware is needed.
For these applications $r(512)$ ranges from 3$\times$ to 197$\times$.
Comparing $r(512)$ to the measured performance ratio of each application we can see that in most cases accelerating the task creation on a big core would not be sufficient for achieving higher task creation rate.
In our experimental evaluation we accelerate task creation in the RTopt and we use the ratio of 16$\times$ which is a relatively small value within this range that we consider realistic to implement in hardware.
%Contrarily, if RTopt is assigned a task to execute, we assume that it executes it 4$\times$ slower than a general purpose in-order core.
The results obtained show the average results among three different traces for each application-input.

\subsection{Homogeneous Multicore Systems}
%\begin{figure}[t]%
%    \label{fig:speedup_homo}
%	\centering
%	%\begin{subfigure}
%	\subfloat [] \includegraphics[width=1.0\textwidth]{figures/speedup_homo.pdf}
%	%\caption{test}
%	%\end{subfigure}
%	%\begin{subfigure}
%	\subfloat [] \includegraphics[width=1.0\textwidth]{figures/speedup_homo2.pdf}
%	%\end{subfigure}
%	\vspace{-0.5cm}
%	\caption{Communication mechanism between master/workers and SRT threads.}
%	\vspace{-0.3cm}
%\end{figure}

\begin{figure}[t]%
	\centering
	\subfloat[]{\label{fig:speedup_homo1}\includegraphics[width=\columnwidth]{figures/speedup_homo.pdf}}
	
	\subfloat[]{\label{fig:speedup_homo2}\includegraphics[width=\columnwidth]{figures/speedup_homo2.pdf}}
	\caption{Speedup of {\proposal} compared to the speedup of Baseline and Baseline+RTopt for each application for systems with 8 up to 512 cores. The average results of (a) show the average among all workloads shown on (a) and (b)}
\end{figure}

Figures~\ref{fig:speedup_homo1} and~\ref{fig:speedup_homo2} show the speedup over one core of three different scenarios: 
\begin{itemize}
	\item \textit{Baseline}: the Nanos++ runtime system, which is the default runtime without using any external hardware support
	\item \textit{Baseline+RTopt}: the Nanos++ runtime system that uses the external hardware as if it is a general purpose core 
	\item \textit{{\proposal}}: our proposed runtime system that takes advantage of the optimized hardware
\end{itemize}
We evaluate these approaches with the TaskSim simulator for systems of 8 up to 512 cores.
In the case of Baseline+RTopt the specialized hardware acts as a slow general purpose core that is additional to the number of cores shown on the x axis.
If this core executes a task creation job, it executes it 16$\times$ faster, but as it is specialized for this, we assume that when a task is executed on this core it is executed 4$\times$ slower than in a general purpose core.
The runtime system in this case does not include our modifications that automatically decouple the task creation step for each task.
The comparison against the Baseline+RTopt is used only to show that the baseline runtime is not capable of effectively utilizing the accelerator. 
In most of the cases having this additional hardware without the appropriate runtime support results in slowdown as the tasks are being executed slower on the special hardware.

Focusing on the average results first, we can observe that {\proposal} constantly improves the baseline and the improvement is increasing as the number of cores is increased, reaching up to 3.1$\times$ improved performance on 512 cores. 
This is because as we increase the number of cores, the task creation overhead becomes more critical part of the execution time and affects performance even more.
So, this becomes the main bottleneck due to which the performance of many applications saturates. 
{\proposal} overcomes it by automatically detecting and moving task creation on the specialized hardware.

Looking in more detail, we can see that for all applications the baseline has a saturation point in speedup.
For example Cholesky256 saturates on 64 cores, while QR512 on 256 cores.
In most cases this saturation in performance comes due to the sequential task creation that is taking place for an important percentage of the execution time (as shown in Figure~\ref{fig:master_thread}).
{\proposal} solves this as it efficiently decouples the task creation code and accelerates it leading to higher speedups.

{\proposal} is effective as it either improves performance or it performs as fast as the baseline (there are no slowdowns). 
The applications that do not benefit (QR512, Ferret, Fluidanimate) are the ones with the highest average per task CPU cycles as shown on Table~\ref{tab.apps}.
Dedup also does not benefit as the per task creation cycles are very low compared to its average task size.
Even if these applications consist of many tasks, the task creation overhead is considered negligible compared to the task cost, so accelerating it does not help much. 
%\begin{figure*}[!t]
%\centering

%\begin{figure}[b]
%\begin{tabular}{@{}c@{}}
%  \includegraphics[width=\textwidth]{figures/speedup_homo.pdf}
%  \caption{}
%  \label{fig:speedup_homo1}
%\end{figure}
%
%\begin{figure}[b]
%  \includegraphics[width=\textwidth]{figures/speedup_homo2.pdf}
%  \caption{}
%  \label{fig:speedup_homo2}
%\end{figure}


This can be verified by the results shown for QR128 workload.
In this case, we use the same input size as QR512 (which is 16K) but we modify the block size, which results in more and smaller tasks.
This not only increases the speedup of the baseline, but also shows even higher speedup when running with {\proposal} reaching very close to the ideal speedup and improving the baseline by 2.3$\times$.
%\begin{figure*}[t]%
%	\centering
%	\includegraphics[width=\textwidth]{figures/speedup_hetero_avg.pdf}
%	\caption{Average speedup among all 11 workloads on heterogeneous simulated systems. The numbers at the bottom of x axis show the total number of cores and the numbers above them show the number of big cores. Results are separated depending on the type of core that executes the master thread: a big or little core.}
%	\label{fig:hetero}%
%	\vspace{-0.3cm}
%\end{figure*}
%\begin{figure}[t]%
%	\centering
%	\includegraphics[width=0.6\columnwidth]{figures/canneal_perf.pdf}
%	\caption{Canneal performance as we modify $r$ }
%	\label{fig:canneal}%
%\end{figure}
\begin{figure}[t]
	\centering
	\includegraphics[width=0.75\columnwidth]{figures/canneal_perf.pdf}
	\caption{Canneal performance as we modify $r$; x-axis shows the number of cores.}
	\label{fig:canneal}
\end{figure}
Modifying the block size for Cholesky, shows the same effect in terms of {\proposal} over baseline improvement.
However, for this application, using the bigger block size of 256 is more efficient as a whole.
Nevertheless, {\proposal} improves the cases that performance saturates and reaches up to 8.5$\times$ improvement for the 256 block-size, and up to 16$\times$ for the 128 block-size.

Blackscholes and Canneal, are applications with very high task creation overheads compared to the task size as shown on Table~\ref{tab.apps}.
This makes them very sensitive to performance degradation due to task creation. 
As a result their performance saturates even with limited core counts of 8 or 16 cores.
These are the ideal cases for using {\proposal} as such bottlenecks are eliminated and performance is improved by 15.9$\times$ and 13.9$\times$ respectively.
However, for Canneal for which the task creation lasts a bit less than half of the task execution time, accelerating it by 16 times is not enough and soon performance saturates at 64 cores. 
In this case, a faster task creation hardware would improve performance even more.
Figure~\ref{fig:canneal} shows how the performance of Canneal is affected when modifying the task creation performance ratio, $r$ between the specialized hardware and general purpose.
Using hardware that performs task creation close to 256$\times$ faster than the general purpose core leads to higher improvements.

Streamcluster has also relatively high task creation overhead compared to the average task cost so improvements are increased as the number of cores is increasing.
{\proposal} reaches up to 7.6$\times$ improvement in this case.

The performance of Bodytrack saturates on 64 cores for the baseline. 
However, it does not approach the ideal speedup as its pipelined parallelization technique introduces significant task dependencies that limit parallelism.
{\proposal} still improves the baseline by up to 37\%.
This improvement is low compared to other benchmarks, firstly because of the nature of the application and secondly because Bodytrack introduces nested parallelism.
With nested parallelism task creation is being spread among cores so it is not becoming a sequential overhead as happens in most of the cases.
Thus, in this case task creation is not as critical to achieve better results.
%correct to look at the CREATE overhead value as this is be parallelized among all cores for the 329\,123 tasks of the application. 


\subsection{Heterogeneous Multicore Systems}

%\begin{figure}[t]%
%	\centering
%	\subfloat[Average speedup among all 11 workloads on heterogeneous simulated systems. The numbers at the bottom of x axis show the total number of cores and the numbers above them show the number of big cores. Results are separated depending on the type of core that executes the master thread: a big or little core.]{\label{fig:hetero}\includegraphics[width=\columnwidth]{figures/speedup_hetero_avg.pdf}}
%	
%	\subfloat[Canneal performance as we modify $r$]{\label{fig:canneal}\includegraphics[width=0.5\columnwidth]{figures/canneal_perf.pdf}}
%\subfloat[Average improvement over baseline]{\label{fig:baseline}\includegraphics[width=0.48\textwidth]{figures/comparison.pdf}}
%	\vspace{-0.3cm}
%	\caption{X-axis of Figures \ref{fig:canneal} and \ref{fig:baseline} shows the number of cores. For each case an RTopt core is used additionally to the number of cores.}
%\end{figure}

\begin{figure*}[t]%
	\centering
	\includegraphics[width=\columnwidth]{figures/speedup_hetero_avg.pdf}
	\caption{Average speedup among all 11 workloads on heterogeneous simulated systems. The numbers at the bottom of x axis show the total number of cores and the numbers above them show the number of big cores. Results are separated depending on the type of core that executes the master thread: a big or little core.}	
	\label{fig:hetero}
\end{figure*}


%	\subfloat[Canneal performance as we modify $r$]{\label{fig:canneal}\includegraphics[width=0.5\columnwidth]{figures/canneal_perf.pdf}}
%\subfloat[Average improvement over baseline]{\label{fig:baseline}\includegraphics[width=0.48\textwidth]{figures/comparison.pdf}}
%	\vspace{-0.3cm}
%	\caption{X-axis of Figures \ref{fig:canneal} and \ref{fig:baseline} shows the number of cores. For each case an RTopt core is used additionally to the number of cores.}
%\end{figure}


%Figure~\ref{fig:hetero} shows the average speedup obtained among the same applications. 
At this stage of the evaluation our system supports two types of general purpose processors, simulating an asymmetric multi-core processor.
The asymmetric system is influenced by the Arm big.LITTLE architecture~\cite{ARM} that consists of big and little cores.
In our simulations, we consider that the big cores are four times faster than the little cores of the system.
This is based on the average measured performance ratio, shown on Table~\ref{tab.apps}, among the 11 workloads used in this evaluation.
%This assumption is based on prior works~\cite{Chronaki:TPDS} that have shown that for most applications the performance ratio ranges from 3.5$\times$ to 4.5$\times$.

In this set-up there are two different ways of executing a task-based application.
The first way is to start the application's execution on a big core of the system and the second way is to start the execution on a little core of the system.
If we use a big core to load the application, then this implies that the master thread of the runtime system (the thread that performs the task creation when running with the baseline) runs on a fast core, thus tasks are created faster than when using a slow core as a starting point.
We evaluate both approaches and compare the results of the baseline runtime and {\proposal}.

Figure~\ref{fig:hetero} plots the average speedup over one little core obtained among all 11 workloads for the Baseline, Baseline+RTopt and {\proposal}.
The chart shows two categories of results on the x axis, separating the cases of the master thread's execution.
The numbers at the bottom of x axis show the total number of cores and the numbers above show the number of big cores.

%The bars represent the average speedup when running with the baseline runtime or with {\proposal} and the line shows the ideal speedup for each configuration.
%The ideal speedup is the speedup that we would obtain if we were running an application in parallel assuming zero runtime overheads and no dependencies between tasks, technically unachievable for the real applications of our evaluation.
%Equation~\ref{eq.ideal} shows how the ideal speedup is computed for our simulated system where the big cores are four times faster than the little cores.
%\begingroup\makeatletter\def\f@size{9}\check@mathfonts
%\begin{equation}
%  \text{$ideal\_speedup(big, little) = big \times 4 + little$}
%\label{eq.ideal}
%\end{equation}
%\endgroup

The results show that moving the master thread from a big to a little core degrades performance of the baseline.
This is because the task creation becomes even slower so the rest of the cores spend more idle time waiting for the tasks to become ready.
{\proposal} improves performance in both cases.
Specifically when master runs on big, the average improvement of {\proposal} reaches 86\%.
When the master thread runs on a little core, {\proposal} improves performance by up to 3.7$\times$. 
This is mainly due to the slowdown caused by the migration of master thread on a little core.
Using {\proposal} on asymmetric systems achieves approximately similar performance regardless of the type of core that the master thread is running. 
This makes our proposal more portable for asymmetric systems as the programmer does not have to be concerned about the type of core that the master thread migrates.


%\subsubsection{Combining TaskGenX with CATS}
\begin{figure*}[t]%
	\centering
	\includegraphics[width=\columnwidth]{figures/TaskGenX+CATS.pdf}
	\caption{Average speedup among 7 dependency synchronized workloads on heterogeneous simulated systems. The numbers at the bottom of x axis show the total number of cores and the numbers above them show the number of big cores.}	
	\label{fig:taskgenx_cats}
\end{figure*}
\begin{table*}[t]
	\begin{center}
		\caption{Evaluated benchmarks and relevant characteristics}
		\label{tab.taskgenx_cats}
		%\resizebox{\textwidth}{!}{%
		\begin{tabular}{|c|c|}
			\hline
			Workload & Avg improvement \\%& {\parbox{50mm}{\centering Per-task CREATE overheads (CPU cycles)}} \\
			\hline
			{\parbox{60mm}{\centering Cholesky 32K 256 (128$\times$128)}} & 0.3\% \\%& 24369 \\                                 
			\hline
			{\parbox{60mm}{\centering Cholesky 32K 128 (256$\times$256)}} & 0.8\% \\%& 31033  \\
			\hline
			{\parbox{60mm}{\centering Cholesky 16K 512 (32$\times$32)}} & 12\% \\%& 19133 \\
			\hline
			QR 16K 512 & 19\% \\%& 20109 \\
			\hline
			QR 16K 128 & 0.5\% \\%& 27620 \\
			\hline
			Bodytrack & 6\% \\%& 10009 \\ 
			%Heat diffusion & Heat &  &  &  &  &  & \\ 
			\hline
			Dedup & 335\% \\%& 1310  \\
			\hline 
			Ferret & 2\% \\%& 42429  \\
			\hline 
		\end{tabular}%}
	\end{center}
\end{table*}			
As shown in this section, TaskGenX effectively improves performance of asymmetric multi-core systems even if the scheduler used is not asymmetry-aware.
In this subsection we combine TaskGenX with the Criticality-Aware Task Scheduler (CATS) that was described in detail in Section~\ref{sec.scheduling.cats} and we comment on the improvements that CATS brings to the current TaskGenX approach.
CATS applies an effective scheduling policy that detects the critical tasks of the application and executes them on the fast cores of the system.
The critical tasks of an application are selected according to their inter-task dependencies. 
This makes CATS more suitable for applications that demonstrate intensive dependencies and TDGs that create long paths.
For applications that their parallelization is mostly based on data parallelism and do not have any inter-task dependencies a scheduler like CATS does not make much sense as it will schedule tasks as randomly as the default BF scheduler.
%In such cases there are no critical tasks, and all tasks are of the same importance since they can execute in parallel without the need of waiting for another task to finish.
For this reason, in this section we omit the results for applications that are barrier synchronized such as blackscholes, canneal, fluidanimate and streamcluster.
After performing experiments, we saw that these applications show no benefit by using CATS because CATS dynamically adapts to the application and gives similar schedules to BF\footnote{All tasks are of same priority so there is no distinction between critical and non-critical. Section~\ref{sec.scheduling.cats} provides detailed description of CATS and how it operates.}. 
Specifically, they present a slowdown around 2\% on average which is due to the task prioritization overhead of CATS.

Figure~\ref{fig:taskgenx_cats} shows the average improvement of TaskGenX+CATS over TaskGenX+BF in bars as well as their speedup in lines.
The average shown on Figure~\ref{fig:taskgenx_cats} is the average of the dependency-synchronized workloads used in this chapter which are: cholesky256, cholesky128, QR512, QR128, bodytrack, dedup and ferret.
As we can see, using an asymmetry-aware dynamic task scheduler on top of TaskGenX further improves the average performance of TaskGenX by up to 46\%.

Moving in more detail, Table~\ref{tab.taskgenx_cats} shows the average improvements obtained from each workload.
TaskGenX+CATS improves TaskGenX+BF by up to 2\% for the two cholesky workloads used here, with the average improvement being 0.3\%. 
%have an improvement from the use of CATS around 3\%. 
Even if cholesky is a dependency intensive application, the benefits of using CATS in combination with TaskGenX are not significant.
This is due to the fact that the specific inputs result at very wide TDGs in which the critical path is not as important.
To verify this fact, we have added the results from a narrow-TDG cholesky workload, which is the input of 16K with 512 block size.
With this workload TaskGenX+CATS achieves up to 22\% improvement over TaskGenX+BF.

The same behavior is observed with QR; the smaller the block size, the wider the TDG, leading to low impact of TaskGenX+CATS for the QR128 input.
TaskGenX+CATS bring improvements up to 61\% when increasing the block size of QR, with the QR512 input.

Bodytrack, Ferret and Dedup also benefit from the use of CATS.
Ferret exhibits high task creation overheads when using CATS, thus its benefit is limited.
TaskGenX compensates by accelerating these task generation overheads but still the improvement cannot surpass 12\% over TaskGenX+BF.
Dedup on the other hand shows very high benefits due to the efficient CATS scheduling.
Dedup is a very good candidate for schedulers like CATS as its TDG structure creates a long path of dependent tasks.
CATS manages to execute these tasks on the big cores resulting in improvements of up to 3.95$\times$.

%However, the dependency-synchronized workloads used in this section are not the ideal cases for CATS.
It is interesting to note that TaskGenX and CATS show somehow contradictory benefits depending on the workload granularity;
TaskGenX is more effective with fine-grained workloads of any synchronization type (barrier or dependency synchronization). 
CATS provides a more workload-specific solution as its performance highly depends on the TDG structure of the application.
However, when combined, TaskGenX and CATS achieve optimal results for asymmetric multi-core systems reaching up to 4$\times$ higher performance over the baseline (no TaskGenX) when the master thread runs on a little core and up to 1.95$\times$ higher performance when the master thread runs on a big core.
%Additionally, CATS benefits from more coarse-grained inputs in order to have the time to detect the new critical tasks while other tasks are being executed.


\subsection{Comparison to Other Approaches}
\begin{figure}[t]
	\centering
	\includegraphics[width=0.75\textwidth]{figures/comparison.pdf}
	\caption{Average improvement of hardware-software proposals over Nanos++ runtime running on each number of cores; x-axis shows the number of cores.}
	\label{fig:compare}
\end{figure}
%Figure~\ref{fig:comparison} shows the average improvement for each core count over the baseline scheduler. 
As we saw earlier, {\proposal} improves the baseline scheduler by up to 6.3$\times$ for 512 cores.
In this section we compare {\proposal} with other approaches.
To do so, we consider the proposals of Carbon~\cite{Carbon}, Task Superscalar~\cite{TaskSS}, Picos++~\cite{Xubin} and Nexus\#~\cite{Nexus}.
We group these proposals based on the part of the runtime activity they are offloading from the CPU.
Carbon and Task Superscalar are runtime-driven meaning that they both accelerate all the runtime and scheduling parts.
The task creation, dependence analysis as well as the scheduling, namely the ready queue manipulation, are transferred to the RTopt with these approaches. 
These overheads are represented on Table~\ref{tab.apps} under ALL.
For the evaluation of these approaches one RTopt is used optimized to accelerate all the runtime activities. 
The second group of related designs that we compare against is the dependencies-driven, which includes approaches like Picos++ and Nexus\#. 
These approaches aim to accelerate only the dependence analysis part of the runtime as well as the scheduling that occurs when a dependency is satisfied.
The RTopt in this case is optimized to accelerate these activities.
For example, when a task finishes execution, and it has produced input for another task, the dependency tracking mechanism is updating the appropriate counters of the reader task and if the task becomes ready, the task is inserted in the ready queue.
The insertion into the ready queue is the scheduling that occurs with the dependence analysis.
These overheads are represented on Table~\ref{tab.apps} under \textit{Deps+Sched}.

%To compare {\proposal} with other systems, we emulate the behaviour of Carbon~\cite{Carbon} and Picos++~\cite{Xubin} in our system.
%In this emulation, we implement Carbon, that originally accelerates scheduling by using hardware queues. 
%To do so we decouple all the possible scheduling overheads and send them for execution by the accelerator. 
%The average per-task scheduling overheads measured are shown on Table~\ref{tab.apps} under SCHED.
%These overheads might seem high compared to the CREATE overheads that {\proposal} accelerates but they are executed among all threads so at the end they do not induce as much delay as task creation does.
%The difference between our Carbon implementation and the original one is that the original one assumes multiple hardware queues, which enables the parallel manipulation by the threads.
%In our case, we are limited to only one queue, as we want to compare an approach that would be as cheap as the {\proposal} approach and use a single hardware component.

Figure~\ref{fig:compare} shows the average improvement in performance for each core count over the performance of the baseline scheduler on the same core count. 
\textit{Runtime} represents the runtime driven approaches and the \textit{Deps} represents the dependencies driven approaches as described above.
X-axis shows the number of general purpose cores; for every core count one additional RTopt core is used.

Accelerating the scheduling with \textit{Runtime}-driven is as efficient as {\proposal} for a limited number of cores, up to 32.
This is because they both accelerate task creation which is an important bottleneck. 
\textit{Deps}-driven approaches on the other hand are not as efficient since in this case the task creation step takes place on the master thread.

Increasing the number of cores, we observe that the improvement of the \textit{Runtime}-driven over the baseline is reduced and stabilized close to 3.2$\times$ while {\proposal} continues to speedup the execution. 
Transferring all parts of the runtime to RTopt with the  \textit{Runtime}-driven approaches, leads to the serialization of the runtime.
Therefore, all scheduling operations (such as enqueue, dequeue of tasks, dependence analysis etc) that typically occur in parallel during runtime are executed sequentially on the RTopt.
Even if RTopt executes these operations faster than a general purpose core, serializing them potentially creates a bottleneck as we increase the number of cores.
{\proposal} does not transfer other runtime activities than the task creation, so it allows scheduling and dependence analysis operations to be performed in a distributed manner.

%We attribute this to the fact that serializing the scheduling operations becomes a bottleneck when increasing the number of cores.
%Scheduling operations (such as enqueue, dequeue of tasks, dependence analysis etc) generally occur in parallel during runtime, so serializing them for systems of up to 32 cores, is efficient.
%With an increased number of cores it is better to perform scheduling in a distributed manner, just as {\proposal} allows.
%
%Scheduling in general (enqueue, dequeue of tasks, dependence analysis etc) occurs in parallel during runtime.
%TaskGenX does not transfer the scheduling to the special hardware. So scheduling parts are executed on each core whenever they occur on the workers. The other approaches that do move the scheduling on the accelerator they serialize it because we assume that the accelerator is centralized. Is this clear? How could we put it clearly in the text?

%\begin{figure}[t]%
%	\centering
%	\includegraphics[width=0.6\textwidth]{figures/comparison.pdf}
%	\caption{Average improvement over baseline. X-axis shows the number of cores. For each case an RTopt core is used additionally to the number of cores.}
%	\label{fig:compare}
%	\vspace{-0.3cm}
%\end{figure}

\textit{Deps} driven approaches go through the same issue of the serialization of the dependency tracking and the scheduling that occurs at the dependence analysis stage.
The reason for the limited performance of \textit{Deps} compared to \textit{Runtime} is that \textit{Deps} does not accelerate any part of the task creation. 
Improvement over the baseline is still significant as performance with \textit{Deps} is improved by up to 1.5$\times$.

{\proposal} is the most efficient software-hardware co-design approach when it comes to highly parallel applications.
On average, it improves the baseline by up to 3.1$\times$ for homogeneous systems and up to 3.7$\times$ for heterogeneous systems.
Compared to other state of the art approaches, {\proposal} is more effective on a large number of cores showing higher performance by 54\% over \textit{Runtime} driven approaches and by 70\% over \textit{Deps} driven approaches.

%\begin{table*}[t]
%	\scriptsize
%	\begin{center}
%		\caption{Evaluated benchmarks and relevant characteristics}
%		\label{tab.taskgenx_cats}
%		%\resizebox{\textwidth}{!}{%
%			\begin{tabular}{|c|c|c|}
%				\hline
%				{\parbox{15mm}{\centering Application}} & 
%				{Min improvement} & 
%				{\parbox{20mm}{\centering Max Improvement}} \\
%				\hline	
%							
%				\multirow{2}{*}{\parbox{20mm}{\centering Cholesky 32K 256}} & 32K 256 & 357\,762   \\                                              
%				 & 32K 128 & 2829058 \\
%				\hline
%				\multirow{2}{*}{\parbox{18mm}{\centering Cholesky 32K 128}} & 16K 512 & 11\,442  \\
%				&  16K 128 & 707\,265 \\
%				\hline
%				QR 512 & native & 488\,202\\
%				\hline
%				QR 128 & & \\
%				\hline
%				Bodytrack & native & 329\,123 \\ 
%				%Heat diffusion & Heat &  &  &  &  &  & \\ 
%				\hline
%				Dedup & native & 20\,248  \\
%				\hline 
%				Ferret & native$\times$2 & 84\,002  \\
%				\hline
%		\end{tabular}%}
%	\end{center}
%\end{table*}
				
	
				