
\chapter{Runtime Overheads Migration}
\label{chapter.taskgenx}
As chip multi-processors (CMPs) are becoming more and more complex, software solutions such as parallel programming models are attracting a lot of attention.
Task-based parallel programming models offer an appealing approach to utilize complex CMPs.
However, the increasing number of cores on modern CMPs is pushing research towards the use of fine grained parallelism.
Task-based programming models need to be able to handle such workloads and offer performance and scalability.
Using specialized hardware for boosting performance of task-based programming models is a common practice in the research community.

This chapter makes the observation that task creation becomes a bottleneck when we execute fine grained parallel applications with task-based programming models.
%From our experiments the time spent on per-task generation ranges from 13 to 39 microseconds.
%From our experiments the average time spent on per-task generation is 19,2 microseconds while the average task time is 51936727.
To quantify this statement, the creation of one task lasts from 0,08\% and can reach up to 38,47\% of one task's execution time\footnote{This percentage depends on the task's characteristics such as its data dependencies}.
Applications whose task creation lasts more than 0,5\% of their per-task duration, task creation becomes a bottleneck as there is no continuous provision of tasks to the available resources.
%
%
%each task creation
%From our experiments task duration lasts from 2.5x of task creation up to 29472x of task creation. 
%Applications where their task duration is less than 165x task duration suffer from task creation overheads. 
As the number of cores increases the time spent generating the tasks of the application is becoming more critical to the entire execution.
To overcome this issue, we propose {\proposal}.
{\proposal} offers a solution for minimizing task creation overheads and relies both on the runtime system and a dedicated hardware.
%We propose \proposal, a task-based programming model that offers a minimalistic approach to runtime overheads acceleration.
On the runtime system side, {\proposal} decouples the task creation from the other runtime activities.
It then transfers this part of the runtime to a specialized hardware.
We draw the requirements for this hardware in order to boost execution of highly parallel applications.
From our evaluation using 11 parallel workloads on both symmetric and asymmetric systems, we obtain performance improvements up to 15$\times$, averaging to 3.1$\times$ over the baseline.
\newpage

%%%%%%%%%%%%%%%%%%%
%%%%%%%%%%%%%%%%%%%
\section{Introduction}
\label{sec:intro}
% <ISPASS17>
Energy efficiency has become the main challenge for future processor designs, motivating prolific research to face the \emph{power wall}. 
Using heterogeneous processing elements is one of the approaches to increase energy efficiency~\cite{CompCores,hetServers}. 
Asymmetric multi-core (AMC) systems is an interesting case of heterogeneous systems to utilize for energy efficiency.
These systems maintain different types of cores that support the same instruction-set architecture. 
The different core types are designed to target different (performance or power) optimization points~\cite{Kumar:ISCA2004,Balakrishnan:ISCA2005,Pangaea}. 

AMCs have been mainly deployed for the mobile market. 
Mobile processors are also utilized in HPC platforms aiming to energy savings~\cite{ARMV8}.
Asymmetric mobile SoCs combine low-power simple cores (\emph{little}) with fast out-of-order cores (\emph{big}) to achieve high performance while keeping power dissipation low.
Another area where AMCs have been successful is the supercomputing market.
The Sunway TaihuLight supercomputer topped the Top500 list in 2016 using AMCs. 
In this setup, big cores, that offer support for speculation to exploit Instruction-Level Parallelism (ILP), run the master tasks such as the OS and runtime system.
Little cores are equipped with wide Single Instruction Multiple Data (SIMD) units and lean pipeline structures for energy efficient execution of compute-intensive code. 

Like in other heterogeneous systems, load balancing and scheduling are fundamental challenges that must be addressed to effectively exploit all the resources in AMC platforms~\cite{Suleman:APLOS2009,Fedorova2009,Greenhalgh2011,Joao:ASPLOS2012,Joao:ISCA2013,ARM4HPC_SC13}. 
Mobile applications rely on multi-programmed workloads to balance the load in the system, while supercomputer applications rely on hand-tuned code to extract maximum performance. 
However, these approaches are not always suitable for general-purpose parallel applications.

In this chapter, we evaluate several execution models on an AMC using the PARSEC benchmark suite~\cite{PARSEC3}. 
This suite includes parallel applications from multiple domains such as finance, computer vision, physics, image processing and video encoding. 
We quantify the performance loss of executing the applications \textit{as-is} on all cores in the system. 
These applications were developed on homogeneous platforms and are bound to suffer from load imbalance on parallel regions that statically distribute the work evenly across cores without considering their performance disparity.

To overcome this matter, we consider two possible solutions at the OS and runtime levels to exploit AMCs effectively.
The first solution delegates scheduling to the OS.
We evaluate the built-in heterogeneity-aware OS scheduler currently used in existing mobile platforms that automatically assigns threads to different core types based on CPU utilization. 
%This approach does not require modifying the application, but is limited for high-utilization multithreaded applications.

The second solution is to transfer the responsibility to the runtime system so it dynamically schedules work to different core types based on work progress and core availability. 
%The advantage is that the runtime system has knowledge of the application structure and parallel work boundaries so it can react with certain level of predictability. 
We evaluate the impact of using an inherently load-balanced execution model such that of task-based programming models. 
Recent examples~\cite{Ayguade:TPDS2009, OpenMP4.0:Manual2013, OmpSs_PPL11, vectorMulticore, Bauer.2012.SC,rollback,Vandierendonck:PACT2011, Vandierendonck:Hyperq,spawn} include clauses to specify inter-task dependencies and remove most barriers which are the major source of load imbalance on AMCs.
Another approach of scheduling in the runtime system is to change the existing statically-scheduled work-sharing constructs for the applications implemented in OpenMP to use dynamic scheduling. 

This chapter provides a comprehensive evaluation of representative parallel applications on a real AMC platform: the Odroid-XU3 development board with Arm big.LITTLE architecture.
We analyze the effectiveness of the aforementioned scheduling solutions in terms of performance, power and energy.
We show why parallel applications are not ready to run on AMCs and how OS and runtime schedulers can overcome these issues depending on the application characteristics.
Further we point out in which aspects the built-in OS scheduler falls short to effectively utilize the AMC.
Finally, we show how the runtime system approach overcomes these issues, and improves the OS and static threading approaches by 13\% and 23\% respectively.

The rest of this chapter is organized as follows: Section~\ref{sec.study.scheduling} provides information on 
scheduling at the OS and runtime system levels.
% while Section~\ref{sec.study.experimental} describes the experimental framework. 
Section~\ref{sec.study.evaluation} shows the performance and energy results and associated insights.% of our experiments. 
Finally, Section~\ref{sec.study.conclusions} concludes this work. 

\iffalse
% <PACT16>
Energy efficiency has become the main 
challenge for future processor designs, motivating prolific research to face the 
\emph{power wall}. Using heterogeneous processing elements is one of the 
approaches to increase energy efficiency. Different types of processors can 
be specialized for different types of computation, such as the combination of 
general-purpose cores with accelerators such as Graphics Processing Units (GPUs). 
Another approach towards heterogeneity is the use of asymmetric multi-cores 
with different types of cores with the same instruction-set architecture. Different core types 
target different performance and power optimization points for energy
efficiency~\cite{Kumar:ISCA2004,Balakrishnan:ISCA2005}. 

Asymmetric multi-cores have been successfully deployed in the mobile market, where 
low-power simple cores (\emph{little}) are combined with 
high-performance out-of-order cores (\emph{big}). Low demand applications
run on little cores for low power operation and prolong battery life. Demanding
applications, such as games, run on the big cores providing high performance
when needed.

Supercomputing is another market where asymmetric multi-cores have been successful. 
The Sunway TaihuLight supercomputer topped the Top500 list in 2016 using asymmetric multi-cores. 
In this setup, big cores, that offer support for speculation and Instruction-Level Parallelism (ILP), run the master tasks such as the OS and runtime system.
%system, as these tasks require support for speculation and Instruction-Level Parallelism (ILP) 
%exploitation of codes with complex control flow.
Little cores are equipped with wide Single Instruction Multiple Data (SIMD) units and lean pipeline structures for energy efficient execution of compute-intensive codes. 

Previous experiences have shown that load balancing and scheduling are fundamental challenges that 
must be addressed to effectively exploit all the resources in these 
platforms~\cite{Suleman:APLOS2009,Fedorova2009,Greenhalgh2011,Joao:ASPLOS2012,Joao:ISCA2013,
ARM4HPC_SC13}. 
Mobile applications rely on multi-programmed workloads to balance the load in the 
system, while supercomputer applications rely on hand-tuned code to extract maximum 
performance. However, these approaches are not always suitable for general-purpose parallel 
applications.
%In a first generation of asymmetric multi-cores, the system could switch from low power to high responding operation modes, activating or de-activating the cluster of big or little cores accordingly~\cite{ARM}. In a second generation of asymmetric multi-core processors, all the cores can run simultaneously to further improve the peak performance of these systems~\cite{samsung}.

%Many researchers are pushing towards building future parallel systems with asymmetric multi-cores~\cite{Suleman:APLOS2009,Fedorova2009, Greenhalgh2011, Joao:ASPLOS2012,Joao:ISCA2013} and even mobile chips~\cite{ARM4HPC_SC13}. However, it is unclear if current parallel applications will benefit from these asymmetric platforms. Load balancing and scheduling are two of the main challenges in utilizing such heterogeneous platforms, as the programmer has to consider them from the very beginning to obtain an efficient parallelization.

%In this paper, we evaluate for the first time the suitability of currently available mobile asymmetric multi-core platforms for general purpose computing. First, we demonstrate that out-of-the-box parallel applications do not run efficiently on asymmetric multi-cores. Fully exploiting the computational power of these processors is challenging as the asymmetry in the system can lead to load imbalance, undermining the scalability of the parallel application. Consequently, only applications that incorporate user-defined load balancing mechanisms can benefit immediately from asymmetric multi-cores.

In this paper, we evaluate several execution models on an asymmetric multi-core
using the PARSEC benchmark suite. This suite includes parallel applications from multiple domains 
such as finance, computer vision, physics, image processing and video encoding. We first quantify 
the performance loss of executing the applications \textit{as-is} on all cores 
in the system. These applications were developed on homogeneous platforms and are bound to suffer from
load imbalance on parallel regions that statically distribute the work
evenly across cores without considering their performance disparity.

Then, we evaluate several solutions at the OS and runtime level that require different
levels of user intervention to exploit asymmetric multi-cores effectively. The first
solution delegates scheduling to the OS. We evaluate the heterogeneity-aware
OS scheduler used in existing mobile platforms that assigns threads to different
core types based on CPU utilization. This requires no modification of the
application, but has limited capability for high-utilization multithreaded applications.

%on an ARM big.LITTLE asymmetric multi-core platform 

%When load-balancing techniques are not included in the original application, we evaluate alternative solutions that, without relying on the programmer, can leverage the opportunities that asymmetric systems offer. In particular, we evaluate a state of the art dynamic scheduler at the Operating System (OS) level that is aware of the characteristics of each core type. This scheduler effectively exploits the system by running high CPU utilization processes on the big cores and low CPU utilization processes on the little cores.

The second solution is to transfer the responsibility to the runtime system so it 
dynamically schedules work to different core types based on work progress and core 
availability. The advantage is that the runtime system has knowledge of the application 
structure and parallel work boundaries so it can react with certain level of predictability. 
We evaluate dynamic scheduling on top of the existing work-sharing constructs in the applications 
with an OpenMP statically-scheduled implementation available. This requires code transformations 
that are straightforward in many cases.

Finally, we evaluate the impact of using an inherently load-balanced execution model such 
that of task-based programming models. 
Recent examples~\cite{Ayguade:TPDS2009, OpenMP4.0:Manual2013, OmpSs_PPL11, Zuckerman:EXADAPT2011, Bauer.2012.SC, Vandierendonck:PACT2011, Vandierendonck:Hyperq} 
include clauses to specify inter-task dependences and remove most barriers which are the major 
source of load imbalance on asymmetric multi-cores.

%and let the runtime system to track dependences between tasks. When these dependences are satisfied, tasks are dynamically scheduled, effectively balancing the workload.

This paper quantifies the effectiveness of these solutions at different levels of the software stack
with a comprehensive evaluation of representative parallel applications on a real 
asymmetric multi-core platform: the Odroid-XU3 development board. This platform features an 
eight-core Samsung Exynos 5422 chip with ARM big.LITTLE architecture with 
four out-of-order Cortex-A15 and four in-order Cortex-A7 cores.

The rest of this document is organized as follows: Section~\ref{sec:background} describes the 
evaluated asymmetric multi-core processor, while Section~\ref{sec:scheduling} offers information on 
dynamic schedulers at the OS and runtime system levels. Section~\ref{sec:experimental} 
describes the experimental framework. Section~\ref{sec:evaluation} shows the performance 
and energy results and associated insights of our experiments. Finally, 
Section~\ref{sec:related} discusses related work and Section~\ref{sec:conclusions} concludes 
this work. 
\fi
%\begin{itemize}
% \item Out-of-the-box applications obtain the best average performance when running only on the aggressive out-of-order cores. Many of these applications are not ready to fully exploit asymmetric multi-cores as they suffer from load imbalance due to the system's heterogeneity. As a result, an average 12\% performance degradation is obtained when using all the cores in the system instead of the four out-of-order cores. 
% \item For the OS scheduler it takes three additional little cores on average to reach the performance obtained with four out-of-order cores. This is observed in most evaluated applications; the addition of little cores to a homogeneous big-core system is degrading performance. Specifically, this slowdown is observed to be 22\% on average when one little core is added to a system that consists of four big cores.
%% applications have 22\% better performance on four big cores compared to their performance on a system with four big and one little cores. 
%When adding four little cores the OS scheduler reduces total execution time by 5.3\% but contrarily to this, it is shown how the runtime system scheduling constantly improves performance by up to 16\%.
%  
%% \item Dynamic scheduling techniques at OS level can turn the tables, reducing the total execution time by 5.3\% when adding four little cores to a system with four big cores. The dynamic scheduler in the runtime system can further improve the final performance by fully utilizing all the resources in the system. This approach reaches an average 13\% speedup, and leads to the most energy efficient solution, as the Energy-Delay Product (EDP) is reduced by 40\% in this configuration.
% \item Moreover, the energy delay product (EDP) results show that the optimal solution taking into account both energy and performance remains the runtime system scheduling.
%%  In systems with a given number of out-of-order cores, adding extra in-order cores can further boost the performance of the application with the appropriate software support (at the application, runtime or OS level). As a result, the energy consumption of parallel applications running on those systems can be reduced by XXX\% on a system with four in-order and four out-of-order cores.
% \item Finally, we evaluate the usefulness of little cores to off-load runtime system activities. Similarly to the assistant core in the IBM Blue Gene Q and the Fujitsu SPARC64 XIfx processors~\cite{BG-Q:HotChips2011, Fujitsu:HotChips2014}, we explore the possibilities of devoting a little or big core to the runtime system activity. In general, we observe that the noise introduced by the runtime system does not degrade the performance of the parallel application. Thus, we can make use of this assistant core to also run user tasks, increasing the final performance of the system.
%\end{itemize}


% </PACT16>

% We describe a set of configurations for our scheduler regarding the work stealing capabilities of the different core types and the flexibility to define a task as critical or non-critical. 
 
% We implement this scheduler in OmpSs and evaluate its effectiveness on different numbers of cores and shares of fast and slow cores on a real system. 
 
% We also evaluate the effectiveness of our scheduler depending on the speed ratio between fast and slow cores using simulation.  The results show that the effectiveness of our scheduler increases with larger numbers of fast cores over slow cores, and with larger differences of performance between fast and slow cores.

%Finally, we provide a set of recommendations on how to configure our scheduler to get the best results depending on the target system size and configuration.



%%%%%%%%%%%%%%%%%%%
%%%%%%%%%%%%%%%%%%%
\section{Background and Motivation}
\label{sec.taskgenx.background}
The OmpSs programming model is a task-based programming model that offers a high level abstraction to the implementation of parallel applications for various homogeneous and heterogeneous architectures~\cite{OmpSs_PPL11,OmpSs}. It enables the annotation of function declarations with the task directive, which declares a task. Every invocation of such a function creates a task that is executed concurrently with other tasks or parallel loops. OmpSs also supports task dependencies and dependency tracking mechanisms~\cite{StarSs}. OmpSs is built with the support of the Mercurium compiler, responsible for the translation of the OmpSs annotation clauses to source code, and the Nanos++ runtime system, responsible for the internal creation and execution of the tasks.

%As a task-based parallel programming model, OmpSs enables the annotation of function declarations with the task directive. If a function is declared as a task, then every invocation of this function creates a task that is executed concurrently with other tasks or parallel loops. The accessible data to each task are the arguments of the function. OmpSs uses the StarSs~\cite{StarSs} dependency tracking mechanisms and each task may be annotated with the \textit{in}, \textit{out}, \textit{inout} clauses. These clauses allow the specification of scalars, arrays and pointers as input, output or input and output data of a task. The implementation of a barrier is supported under the \textit{taskwait} clause, and it can also be used with the addition of the \textit{on} clause, to declare a barrier for the group of tasks that produce a specific piece of data. These original OmpSs features can now be found in OpenMP 4.0~\cite{OpenMP}.

Nanos++ is an environment that serves as the runtime platform of OmpSs. It provides device support for heterogeneity and includes different plug-ins for implementations of schedulers, throttling policies, barriers, dependency tracking mechanisms, work-sharing and instrumentation. This design allows to maintain the runtime features by adding or removing plug-ins, facilitating the implementation of a new scheduler, or the support of a new architecture.

The implementations of the different scheduling policies in Nanos++ perform various actions on the states of the tasks. A task is \textit{created} if a call to this task is discovered but it is waiting until all its inputs are produced by previous tasks. When all the input dependencies are satisfied, the task becomes \textit{ready}. The ready tasks of the application at a given point in time are inserted in the \textit{ready queues} as stated by the scheduling policy. Ready queues can be thread-private or shared among threads. When a thread becomes idle, the scheduling policy picks a task from the ready queues for that thread to execute. The default OmpSs scheduler employs a \textit{breadth-first} policy~(BF)~\cite{Duran_schedulers_08} and implements a single first-in-first-out ready queue shared among all threads. When a task is ready, it is inserted in the tail of the ready queue and when a core becomes available, it retrieves a task from the head of the queue. BF does not differentiate among core types and assigns tasks in a first-come-first-served basis. We use this scheduler as our baseline.

The Nanos++ internal data structures support task prioritization. The task priority is an integer field inside the task descriptor that rates the importance of the task. If the scheduling policy supports priorities, the ready queues are implemented as \textit{priority queues}. In a priority queue, tasks are sorted in a decreasing order of their priority. The insertion in a priority queue is always ordered and the removal of a task is always from the head of the queue, i.e., the task with the highest priority. The priority of a task can be either set in user code, by using the \textit{priority} clause, which accepts an integer priority value or expression, or dynamically  by the scheduling policy, as is described in the next section.

%The default OmpSs scheduler employs a \textit{breadth-first} policy~(BF)~\cite{Duran_schedulers_08}. The BF scheduler implements a single first-in-first-out ready queue shared among all threads. When a task is ready, it is inserted in the tail of the ready queue and when a core becomes available, it retrieves a task from the head of the queue. Tasks are ordered according to their ready time: the earliest ready task resides at the head of the queue. Since the ready queue is shared, there is no need for work stealing and the load is balanced automatically. BF does not differentiate among core types and assigns tasks in a first-come-first-served basis. We use this scheduler as a baseline for the evaluation.

%%%%%%%%%%%%%%%%%%%
%%%%%%%%%%%%%%%%%%%
\section{Task Generation Express (TaskGenX)}
\label{sec.taskgenx.ram}
\input{TaskGenX/03-our_work}

%\section{Hardware requirements}}
%\label{sec:hw}
\input{TaskGenX/hw}
%%%%%%%%%%%%%%%%%%%
%%%%%%%%%%%%%%%%%%%

\input{TaskGenX/04-experimental}

%%%%%%%%%%%%%%%%%%%
%%%%%%%%%%%%%%%%%%%
\section{Evaluation}
\label{sec.taskgenx.evaluation}
\subsection{Methodology}
\label{sec.taskgenx.methodology}
\begin{table*}[t]
	\scriptsize
	\begin{center}
		\caption{Evaluated benchmarks and relevant characteristics}
		\label{tab.apps}
		\resizebox{\textwidth}{!}{%
			\begin{tabular}{|c|c|c|c|c|c|c|c|c|}
				\hline
				\multirow{3}{*}{\parbox{15mm}{\centering Application}} & 
				\multirow{3}{*}{Problem size} & 
				\multirow{3}{*}{\parbox{10mm}{\centering \#Tasks}} & 
				\multirow{3}{*}{\parbox{17mm}{\centering Avg task CPU cycles (thousands)}} & 
				\multicolumn{3}{|c|}{\parbox{22mm}{\centering Per task overheads (CPU cycles)}} & & \\
				\cline{5-7}
				& & & & \multirow{2}{*}{\parbox{10mm}{\centering Create}} & \multirow{2}{*}{\parbox{9mm}{\centering All}} & \multirow{2}{*}{\parbox{11mm}{\centering Deps + Sched}} & \multirow{2}{*}{\parbox{15mm}{\centering Measured perf. ratio}} &
				\multirow{2}{*}{\parbox{10mm}{\centering $r(512)$}} \\
				& & & & & & & &  \\ %\hhline{~~~~~~}
				\hline
				
				\multirow{2}{*}{\parbox{20mm}{\centering Cholesky factorization}} & 32K 256 & 357\,762  & 753 & 15221 &  73286 &  58065 &  \multirow{2}{*}{\parbox{9mm}{\centering 3.5}} & 10.34 \\                                              
				& 32K 128 & 2829058 & 110 & 17992 &  58820 &  40828 & & 83.74 \\
				%& 32$\times$32 blocks of 512$\times$512 floats & 5984 & & 1\,551\,322 & 104.76 &  238.02 &  194.28  & \\ 
				\hline{}
				\multirow{2}{*}{\parbox{18mm}{\centering QR factorization}} & 16K 512 & 11\,442 & 518\,570  & 17595 & 63008 &   45413 & \multirow{2}{*}{\parbox{9mm}{\centering 6.8}} & 0.01 \\
				&  16K 128 & 707\,265 & 3\,558 & 21642 & 60777 & 39135 & & 3.11 \\
				\hline
				Blackscholes & native & 488\,202 & 348  &   29141  &  85438 &  56297 & 2.3 & 42.87   \\
				\hline
				Bodytrack & native & 329\,123 & 383 &  9\,505 &  18979 & 9474 & 4.2 & 12.70   \\ 
				%Heat diffusion & Heat &  &  &  &  &  & \\ 
				\hline
				Canneal & native & 3\,072\,002 & 67 & 25781 & 50094 &  24313 & 2.0 & 197.01   \\
				\hline
				Dedup & native & 20\,248 & 1\,532 & 1294 & 9647 &  8353 & 2.7 & 0.43   \\
				\hline 
				Ferret & native$\times$2 & 84\,002 & 29\,088 & 38913 & 98457 &  59544 & 3.6 & 0.68   \\
				\hline
				Fluidanimate & native & 128\,502 & 16\,734 & 30210 & 94079 &  64079 & 3.3 & 0.91   \\
				\hline
				Streamcluster & native & 3\,184\,654 & 161 & 6892 & 13693 &  6801 & 3.5 & 21.91   \\
				\hline
		\end{tabular}}
	\end{center}
	\vspace{-0.4cm}
\end{table*}

\subsubsection{Applications}
%\begin{itemize}
%\item Blackscholes
%\item Cholesky
%\item Canneal
%\item Fluidanimate
%\item QR Factorization
%\item Bodytrack
%\item Streamcluster
%\end{itemize}
Table~\ref{tab.apps} shows the evaluated applications, the input sizes used, and their characteristics. 
All applications are implemented using the OmpSs programming model~\cite{OpenMP4.0:Manual2015}. 
We obtain Cholesky and QR from the BAR repository~\cite{BAR} and we use the implementations of the rest of the benchmarks from the PARSECSs suite~\cite{Chasapis:TACO2016}.
More information about these applications can be found in~\cite{Chasapis:TACO2016} and~\cite{Chronaki:ICS2015}.
As the number of cores in SoCs is increasing, so does the need of available task parallelism~\cite{Sanchez:2010}. 
We choose the input sizes of the applications so that they are realistic cases for real life HPC applications and at the same time, create enough fine-grained tasks to feed up to 512 cores.
The number of tasks per application and input as well as the average per-task CPU cycles can be found on Table~\ref{tab.apps}.





\subsubsection{Simulation}
\label{sec:experimental:simulation}
To evaluate {\proposal} we make use of the trace-driven TaskSim simulator~\cite{AbstrLevels_TACO12,MUSA} which is described in section~\ref{sec.background.simulation}. 
%Added in the background section:
%TaskSim is a trace driven simulator, that supports the specification of homogeneous or heterogeneous systems with many cores. 
%The tracing overhead of the simulator is less than 10\% and the simulation is accurate as long as there is no contention in the shared memory resources on a real system~\cite{MUSA}.
%By default, TaskSim allows the specification of the amount of cores and supports up to two core types in the case of heterogeneous asymmetric systems. 
%This is done by specifying the number of cores of each type and their difference in performance between the different types (performance ratio) in the TaskSim configuration file.

We evaluate the effectiveness of TaskGenX on both symmetric and asymmetric platforms with the number of cores varying from 8 to 512.
In the case of asymmetric systems, we simulate the behavior of an Arm big.LITTLE architecture~\cite{ARM}.
To set the correct performance ratio between big and little cores, we measure the sequential execution time of each application on a real Arm big.LITTLE platform when running on a little and on a big core. 
We use the Hardkernel Odroid~XU3 board that includes a Samsung Exynos 5422 chip with Arm big.LITTLE.
The big cores run at 1.6GHz and the little cores at 800MHz.
%We compare its performance when they run on a little and on a big core.
Table~\ref{tab.apps} shows the measured performance ratio for each case.
The average performance ratio among our 11 workloads is 3.8.
Thus in the specification of the asymmetric systems we use as performance ratio the value 4.

%Added in the background:
%To simulate our approaches using TaskSim we first run each application/input in the TaskSim trace generation mode.
%This mode enables the online tracking of task duration and synchronization overheads and stores them in a trace file. 
%To perform the simulation, TaskSim uses the information stored in the trace file and executes the application by providing this information to the runtime system.
%For our experiments we generate three trace files for each application/input combination on a Genuine Intel 16-core machine running at 2.60GHz.

For the needs of this hardware-software co-design, we modify TaskSim so that it features one extra hardware accelerator (per multi-core) responsible for the fast task creation (the RTopt).
Apart from the task duration time, our modified simulator tracks the duration of the runtime overheads.
These overheads include: (a) task creation, (b) dependencies resolution, and (c) scheduling.
The RTopt core is optimized to execute task creation faster than the general purpose cores; 
to determine how much faster a task creation job is executed we use the analysis performed in Section~\ref{sec:hw_req}.

Using Equation~\ref{eq.create}, we compute the $C_{opt}(x)$ for each application according to their average task CPU cycles from Table~\ref{tab.apps} for $x=512$ cores.
$C_{gp}$ is the cost of task creation when it is performed on a general purpose core, namely the \textit{Create} column shown on Table~\ref{tab.apps}.
To have optimal results for each application on systems up to 512 cores, $C_{gp}$ needs to be reduced to $C_{opt}(512)$.
Thus the specialized hardware accelerator needs to perform task creation with a ratio $r(512) = C_{gp}/C_{opt}(512) \times$ faster than a general purpose core.

We compute $r(512)$ for each application shown on Table~\ref{tab.apps}. We observe that for the applications with a large number of per-task CPU cycles and relatively small \textit{Create} cycles (QR512, Dedup, Ferret, Fluidanimate), $r(512)$ is very close to zero, meaning that the task creation cost ($C_{gp}$) is already small enough for optimal task creation without the need of a faster hardware accelerator.
For the rest of the applications, RTopt needs to be more efficient. %powerful hardware is needed.
For these applications $r(512)$ ranges from 3$\times$ to 197$\times$.
Comparing $r(512)$ to the measured performance ratio of each application we can see that in most cases accelerating the task creation on a big core would not be sufficient for achieving higher task creation rate.
In our experimental evaluation we accelerate task creation in the RTopt and we use the ratio of 16$\times$ which is a relatively small value within this range that we consider realistic to implement in hardware.
%Contrarily, if RTopt is assigned a task to execute, we assume that it executes it 4$\times$ slower than a general purpose in-order core.
The results obtained show the average results among three different traces for each application-input.

\subsection{Homogeneous Multicore Systems}
%\begin{figure}[t]%
%    \label{fig:speedup_homo}
%	\centering
%	%\begin{subfigure}
%	\subfloat [] \includegraphics[width=1.0\textwidth]{figures/speedup_homo.pdf}
%	%\caption{test}
%	%\end{subfigure}
%	%\begin{subfigure}
%	\subfloat [] \includegraphics[width=1.0\textwidth]{figures/speedup_homo2.pdf}
%	%\end{subfigure}
%	\vspace{-0.5cm}
%	\caption{Communication mechanism between master/workers and SRT threads.}
%	\vspace{-0.3cm}
%\end{figure}

\begin{figure}[t]%
	\centering
	\subfloat[]{\label{fig:speedup_homo1}\includegraphics[width=\columnwidth]{figures/speedup_homo.pdf}}
	
	\subfloat[]{\label{fig:speedup_homo2}\includegraphics[width=\columnwidth]{figures/speedup_homo2.pdf}}
	\caption{Speedup of {\proposal} compared to the speedup of Baseline and Baseline+RTopt for each application for systems with 8 up to 512 cores. The average results of (a) show the average among all workloads shown on (a) and (b)}
\end{figure}

Figures~\ref{fig:speedup_homo1} and~\ref{fig:speedup_homo2} show the speedup over one core of three different scenarios: 
\begin{itemize}
	\item \textit{Baseline}: the Nanos++ runtime system, which is the default runtime without using any external hardware support
	\item \textit{Baseline+RTopt}: the Nanos++ runtime system that uses the external hardware as if it is a general purpose core 
	\item \textit{{\proposal}}: our proposed runtime system that takes advantage of the optimized hardware
\end{itemize}
We evaluate these approaches with the TaskSim simulator for systems of 8 up to 512 cores.
In the case of Baseline+RTopt the specialized hardware acts as a slow general purpose core that is additional to the number of cores shown on the x axis.
If this core executes a task creation job, it executes it 16$\times$ faster, but as it is specialized for this, we assume that when a task is executed on this core it is executed 4$\times$ slower than in a general purpose core.
The runtime system in this case does not include our modifications that automatically decouple the task creation step for each task.
The comparison against the Baseline+RTopt is used only to show that the baseline runtime is not capable of effectively utilizing the accelerator. 
In most of the cases having this additional hardware without the appropriate runtime support results in slowdown as the tasks are being executed slower on the special hardware.

Focusing on the average results first, we can observe that {\proposal} constantly improves the baseline and the improvement is increasing as the number of cores is increased, reaching up to 3.1$\times$ improved performance on 512 cores. 
This is because as we increase the number of cores, the task creation overhead becomes more critical part of the execution time and affects performance even more.
So, this becomes the main bottleneck due to which the performance of many applications saturates. 
{\proposal} overcomes it by automatically detecting and moving task creation on the specialized hardware.

Looking in more detail, we can see that for all applications the baseline has a saturation point in speedup.
For example Cholesky256 saturates on 64 cores, while QR512 on 256 cores.
In most cases this saturation in performance comes due to the sequential task creation that is taking place for an important percentage of the execution time (as shown in Figure~\ref{fig:master_thread}).
{\proposal} solves this as it efficiently decouples the task creation code and accelerates it leading to higher speedups.

{\proposal} is effective as it either improves performance or it performs as fast as the baseline (there are no slowdowns). 
The applications that do not benefit (QR512, Ferret, Fluidanimate) are the ones with the highest average per task CPU cycles as shown on Table~\ref{tab.apps}.
Dedup also does not benefit as the per task creation cycles are very low compared to its average task size.
Even if these applications consist of many tasks, the task creation overhead is considered negligible compared to the task cost, so accelerating it does not help much. 
%\begin{figure*}[!t]
%\centering

%\begin{figure}[b]
%\begin{tabular}{@{}c@{}}
%  \includegraphics[width=\textwidth]{figures/speedup_homo.pdf}
%  \caption{}
%  \label{fig:speedup_homo1}
%\end{figure}
%
%\begin{figure}[b]
%  \includegraphics[width=\textwidth]{figures/speedup_homo2.pdf}
%  \caption{}
%  \label{fig:speedup_homo2}
%\end{figure}


This can be verified by the results shown for QR128 workload.
In this case, we use the same input size as QR512 (which is 16K) but we modify the block size, which results in more and smaller tasks.
This not only increases the speedup of the baseline, but also shows even higher speedup when running with {\proposal} reaching very close to the ideal speedup and improving the baseline by 2.3$\times$.
%\begin{figure*}[t]%
%	\centering
%	\includegraphics[width=\textwidth]{figures/speedup_hetero_avg.pdf}
%	\caption{Average speedup among all 11 workloads on heterogeneous simulated systems. The numbers at the bottom of x axis show the total number of cores and the numbers above them show the number of big cores. Results are separated depending on the type of core that executes the master thread: a big or little core.}
%	\label{fig:hetero}%
%	\vspace{-0.3cm}
%\end{figure*}
%\begin{figure}[t]%
%	\centering
%	\includegraphics[width=0.6\columnwidth]{figures/canneal_perf.pdf}
%	\caption{Canneal performance as we modify $r$ }
%	\label{fig:canneal}%
%\end{figure}
\begin{figure}[t]
	\centering
	\includegraphics[width=0.75\columnwidth]{figures/canneal_perf.pdf}
	\caption{Canneal performance as we modify $r$; x-axis shows the number of cores.}
	\label{fig:canneal}
\end{figure}
Modifying the block size for Cholesky, shows the same effect in terms of {\proposal} over baseline improvement.
However, for this application, using the bigger block size of 256 is more efficient as a whole.
Nevertheless, {\proposal} improves the cases that performance saturates and reaches up to 8.5$\times$ improvement for the 256 block-size, and up to 16$\times$ for the 128 block-size.

Blackscholes and Canneal, are applications with very high task creation overheads compared to the task size as shown on Table~\ref{tab.apps}.
This makes them very sensitive to performance degradation due to task creation. 
As a result their performance saturates even with limited core counts of 8 or 16 cores.
These are the ideal cases for using {\proposal} as such bottlenecks are eliminated and performance is improved by 15.9$\times$ and 13.9$\times$ respectively.
However, for Canneal for which the task creation lasts a bit less than half of the task execution time, accelerating it by 16 times is not enough and soon performance saturates at 64 cores. 
In this case, a faster task creation hardware would improve performance even more.
Figure~\ref{fig:canneal} shows how the performance of Canneal is affected when modifying the task creation performance ratio, $r$ between the specialized hardware and general purpose.
Using hardware that performs task creation close to 256$\times$ faster than the general purpose core leads to higher improvements.

Streamcluster has also relatively high task creation overhead compared to the average task cost so improvements are increased as the number of cores is increasing.
{\proposal} reaches up to 7.6$\times$ improvement in this case.

The performance of Bodytrack saturates on 64 cores for the baseline. 
However, it does not approach the ideal speedup as its pipelined parallelization technique introduces significant task dependencies that limit parallelism.
{\proposal} still improves the baseline by up to 37\%.
This improvement is low compared to other benchmarks, firstly because of the nature of the application and secondly because Bodytrack introduces nested parallelism.
With nested parallelism task creation is being spread among cores so it is not becoming a sequential overhead as happens in most of the cases.
Thus, in this case task creation is not as critical to achieve better results.
%correct to look at the CREATE overhead value as this is be parallelized among all cores for the 329\,123 tasks of the application. 


\subsection{Heterogeneous Multicore Systems}

%\begin{figure}[t]%
%	\centering
%	\subfloat[Average speedup among all 11 workloads on heterogeneous simulated systems. The numbers at the bottom of x axis show the total number of cores and the numbers above them show the number of big cores. Results are separated depending on the type of core that executes the master thread: a big or little core.]{\label{fig:hetero}\includegraphics[width=\columnwidth]{figures/speedup_hetero_avg.pdf}}
%	
%	\subfloat[Canneal performance as we modify $r$]{\label{fig:canneal}\includegraphics[width=0.5\columnwidth]{figures/canneal_perf.pdf}}
%\subfloat[Average improvement over baseline]{\label{fig:baseline}\includegraphics[width=0.48\textwidth]{figures/comparison.pdf}}
%	\vspace{-0.3cm}
%	\caption{X-axis of Figures \ref{fig:canneal} and \ref{fig:baseline} shows the number of cores. For each case an RTopt core is used additionally to the number of cores.}
%\end{figure}

\begin{figure*}[t]%
	\centering
	\includegraphics[width=\columnwidth]{figures/speedup_hetero_avg.pdf}
	\caption{Average speedup among all 11 workloads on heterogeneous simulated systems. The numbers at the bottom of x axis show the total number of cores and the numbers above them show the number of big cores. Results are separated depending on the type of core that executes the master thread: a big or little core.}	
	\label{fig:hetero}
\end{figure*}


%	\subfloat[Canneal performance as we modify $r$]{\label{fig:canneal}\includegraphics[width=0.5\columnwidth]{figures/canneal_perf.pdf}}
%\subfloat[Average improvement over baseline]{\label{fig:baseline}\includegraphics[width=0.48\textwidth]{figures/comparison.pdf}}
%	\vspace{-0.3cm}
%	\caption{X-axis of Figures \ref{fig:canneal} and \ref{fig:baseline} shows the number of cores. For each case an RTopt core is used additionally to the number of cores.}
%\end{figure}


%Figure~\ref{fig:hetero} shows the average speedup obtained among the same applications. 
At this stage of the evaluation our system supports two types of general purpose processors, simulating an asymmetric multi-core processor.
The asymmetric system is influenced by the Arm big.LITTLE architecture~\cite{ARM} that consists of big and little cores.
In our simulations, we consider that the big cores are four times faster than the little cores of the system.
This is based on the average measured performance ratio, shown on Table~\ref{tab.apps}, among the 11 workloads used in this evaluation.
%This assumption is based on prior works~\cite{Chronaki:TPDS} that have shown that for most applications the performance ratio ranges from 3.5$\times$ to 4.5$\times$.

In this set-up there are two different ways of executing a task-based application.
The first way is to start the application's execution on a big core of the system and the second way is to start the execution on a little core of the system.
If we use a big core to load the application, then this implies that the master thread of the runtime system (the thread that performs the task creation when running with the baseline) runs on a fast core, thus tasks are created faster than when using a slow core as a starting point.
We evaluate both approaches and compare the results of the baseline runtime and {\proposal}.

Figure~\ref{fig:hetero} plots the average speedup over one little core obtained among all 11 workloads for the Baseline, Baseline+RTopt and {\proposal}.
The chart shows two categories of results on the x axis, separating the cases of the master thread's execution.
The numbers at the bottom of x axis show the total number of cores and the numbers above show the number of big cores.

%The bars represent the average speedup when running with the baseline runtime or with {\proposal} and the line shows the ideal speedup for each configuration.
%The ideal speedup is the speedup that we would obtain if we were running an application in parallel assuming zero runtime overheads and no dependencies between tasks, technically unachievable for the real applications of our evaluation.
%Equation~\ref{eq.ideal} shows how the ideal speedup is computed for our simulated system where the big cores are four times faster than the little cores.
%\begingroup\makeatletter\def\f@size{9}\check@mathfonts
%\begin{equation}
%  \text{$ideal\_speedup(big, little) = big \times 4 + little$}
%\label{eq.ideal}
%\end{equation}
%\endgroup

The results show that moving the master thread from a big to a little core degrades performance of the baseline.
This is because the task creation becomes even slower so the rest of the cores spend more idle time waiting for the tasks to become ready.
{\proposal} improves performance in both cases.
Specifically when master runs on big, the average improvement of {\proposal} reaches 86\%.
When the master thread runs on a little core, {\proposal} improves performance by up to 3.7$\times$. 
This is mainly due to the slowdown caused by the migration of master thread on a little core.
Using {\proposal} on asymmetric systems achieves approximately similar performance regardless of the type of core that the master thread is running. 
This makes our proposal more portable for asymmetric systems as the programmer does not have to be concerned about the type of core that the master thread migrates.


%\subsubsection{Combining TaskGenX with CATS}
\begin{figure*}[t]%
	\centering
	\includegraphics[width=\columnwidth]{figures/TaskGenX+CATS.pdf}
	\caption{Average speedup among 7 dependency synchronized workloads on heterogeneous simulated systems. The numbers at the bottom of x axis show the total number of cores and the numbers above them show the number of big cores.}	
	\label{fig:taskgenx_cats}
\end{figure*}
\begin{table*}[t]
	\begin{center}
		\caption{Evaluated benchmarks and relevant characteristics}
		\label{tab.taskgenx_cats}
		%\resizebox{\textwidth}{!}{%
		\begin{tabular}{|c|c|}
			\hline
			Workload & Avg improvement \\%& {\parbox{50mm}{\centering Per-task CREATE overheads (CPU cycles)}} \\
			\hline
			{\parbox{60mm}{\centering Cholesky 32K 256 (128$\times$128)}} & 0.3\% \\%& 24369 \\                                 
			\hline
			{\parbox{60mm}{\centering Cholesky 32K 128 (256$\times$256)}} & 0.8\% \\%& 31033  \\
			\hline
			{\parbox{60mm}{\centering Cholesky 16K 512 (32$\times$32)}} & 12\% \\%& 19133 \\
			\hline
			QR 16K 512 & 19\% \\%& 20109 \\
			\hline
			QR 16K 128 & 0.5\% \\%& 27620 \\
			\hline
			Bodytrack & 6\% \\%& 10009 \\ 
			%Heat diffusion & Heat &  &  &  &  &  & \\ 
			\hline
			Dedup & 335\% \\%& 1310  \\
			\hline 
			Ferret & 2\% \\%& 42429  \\
			\hline 
		\end{tabular}%}
	\end{center}
\end{table*}			
As shown in this section, TaskGenX effectively improves performance of asymmetric multi-core systems even if the scheduler used is not asymmetry-aware.
In this subsection we combine TaskGenX with the Criticality-Aware Task Scheduler (CATS) that was described in detail in Section~\ref{sec.scheduling.cats} and we comment on the improvements that CATS brings to the current TaskGenX approach.
CATS applies an effective scheduling policy that detects the critical tasks of the application and executes them on the fast cores of the system.
The critical tasks of an application are selected according to their inter-task dependencies. 
This makes CATS more suitable for applications that demonstrate intensive dependencies and TDGs that create long paths.
For applications that their parallelization is mostly based on data parallelism and do not have any inter-task dependencies a scheduler like CATS does not make much sense as it will schedule tasks as randomly as the default BF scheduler.
%In such cases there are no critical tasks, and all tasks are of the same importance since they can execute in parallel without the need of waiting for another task to finish.
For this reason, in this section we omit the results for applications that are barrier synchronized such as blackscholes, canneal, fluidanimate and streamcluster.
After performing experiments, we saw that these applications show no benefit by using CATS because CATS dynamically adapts to the application and gives similar schedules to BF\footnote{All tasks are of same priority so there is no distinction between critical and non-critical. Section~\ref{sec.scheduling.cats} provides detailed description of CATS and how it operates.}. 
Specifically, they present a slowdown around 2\% on average which is due to the task prioritization overhead of CATS.

Figure~\ref{fig:taskgenx_cats} shows the average improvement of TaskGenX+CATS over TaskGenX+BF in bars as well as their speedup in lines.
The average shown on Figure~\ref{fig:taskgenx_cats} is the average of the dependency-synchronized workloads used in this chapter which are: cholesky256, cholesky128, QR512, QR128, bodytrack, dedup and ferret.
As we can see, using an asymmetry-aware dynamic task scheduler on top of TaskGenX further improves the average performance of TaskGenX by up to 46\%.

Moving in more detail, Table~\ref{tab.taskgenx_cats} shows the average improvements obtained from each workload.
TaskGenX+CATS improves TaskGenX+BF by up to 2\% for the two cholesky workloads used here, with the average improvement being 0.3\%. 
%have an improvement from the use of CATS around 3\%. 
Even if cholesky is a dependency intensive application, the benefits of using CATS in combination with TaskGenX are not significant.
This is due to the fact that the specific inputs result at very wide TDGs in which the critical path is not as important.
To verify this fact, we have added the results from a narrow-TDG cholesky workload, which is the input of 16K with 512 block size.
With this workload TaskGenX+CATS achieves up to 22\% improvement over TaskGenX+BF.

The same behavior is observed with QR; the smaller the block size, the wider the TDG, leading to low impact of TaskGenX+CATS for the QR128 input.
TaskGenX+CATS bring improvements up to 61\% when increasing the block size of QR, with the QR512 input.

Bodytrack, Ferret and Dedup also benefit from the use of CATS.
Ferret exhibits high task creation overheads when using CATS, thus its benefit is limited.
TaskGenX compensates by accelerating these task generation overheads but still the improvement cannot surpass 12\% over TaskGenX+BF.
Dedup on the other hand shows very high benefits due to the efficient CATS scheduling.
Dedup is a very good candidate for schedulers like CATS as its TDG structure creates a long path of dependent tasks.
CATS manages to execute these tasks on the big cores resulting in improvements of up to 3.95$\times$.

%However, the dependency-synchronized workloads used in this section are not the ideal cases for CATS.
It is interesting to note that TaskGenX and CATS show somehow contradictory benefits depending on the workload granularity;
TaskGenX is more effective with fine-grained workloads of any synchronization type (barrier or dependency synchronization). 
CATS provides a more workload-specific solution as its performance highly depends on the TDG structure of the application.
However, when combined, TaskGenX and CATS achieve optimal results for asymmetric multi-core systems reaching up to 4$\times$ higher performance over the baseline (no TaskGenX) when the master thread runs on a little core and up to 1.95$\times$ higher performance when the master thread runs on a big core.
%Additionally, CATS benefits from more coarse-grained inputs in order to have the time to detect the new critical tasks while other tasks are being executed.


\subsection{Comparison to Other Approaches}
\begin{figure}[t]
	\centering
	\includegraphics[width=0.75\textwidth]{figures/comparison.pdf}
	\caption{Average improvement of hardware-software proposals over Nanos++ runtime running on each number of cores; x-axis shows the number of cores.}
	\label{fig:compare}
\end{figure}
%Figure~\ref{fig:comparison} shows the average improvement for each core count over the baseline scheduler. 
As we saw earlier, {\proposal} improves the baseline scheduler by up to 6.3$\times$ for 512 cores.
In this section we compare {\proposal} with other approaches.
To do so, we consider the proposals of Carbon~\cite{Carbon}, Task Superscalar~\cite{TaskSS}, Picos++~\cite{Xubin} and Nexus\#~\cite{Nexus}.
We group these proposals based on the part of the runtime activity they are offloading from the CPU.
Carbon and Task Superscalar are runtime-driven meaning that they both accelerate all the runtime and scheduling parts.
The task creation, dependence analysis as well as the scheduling, namely the ready queue manipulation, are transferred to the RTopt with these approaches. 
These overheads are represented on Table~\ref{tab.apps} under ALL.
For the evaluation of these approaches one RTopt is used optimized to accelerate all the runtime activities. 
The second group of related designs that we compare against is the dependencies-driven, which includes approaches like Picos++ and Nexus\#. 
These approaches aim to accelerate only the dependence analysis part of the runtime as well as the scheduling that occurs when a dependency is satisfied.
The RTopt in this case is optimized to accelerate these activities.
For example, when a task finishes execution, and it has produced input for another task, the dependency tracking mechanism is updating the appropriate counters of the reader task and if the task becomes ready, the task is inserted in the ready queue.
The insertion into the ready queue is the scheduling that occurs with the dependence analysis.
These overheads are represented on Table~\ref{tab.apps} under \textit{Deps+Sched}.

%To compare {\proposal} with other systems, we emulate the behaviour of Carbon~\cite{Carbon} and Picos++~\cite{Xubin} in our system.
%In this emulation, we implement Carbon, that originally accelerates scheduling by using hardware queues. 
%To do so we decouple all the possible scheduling overheads and send them for execution by the accelerator. 
%The average per-task scheduling overheads measured are shown on Table~\ref{tab.apps} under SCHED.
%These overheads might seem high compared to the CREATE overheads that {\proposal} accelerates but they are executed among all threads so at the end they do not induce as much delay as task creation does.
%The difference between our Carbon implementation and the original one is that the original one assumes multiple hardware queues, which enables the parallel manipulation by the threads.
%In our case, we are limited to only one queue, as we want to compare an approach that would be as cheap as the {\proposal} approach and use a single hardware component.

Figure~\ref{fig:compare} shows the average improvement in performance for each core count over the performance of the baseline scheduler on the same core count. 
\textit{Runtime} represents the runtime driven approaches and the \textit{Deps} represents the dependencies driven approaches as described above.
X-axis shows the number of general purpose cores; for every core count one additional RTopt core is used.

Accelerating the scheduling with \textit{Runtime}-driven is as efficient as {\proposal} for a limited number of cores, up to 32.
This is because they both accelerate task creation which is an important bottleneck. 
\textit{Deps}-driven approaches on the other hand are not as efficient since in this case the task creation step takes place on the master thread.

Increasing the number of cores, we observe that the improvement of the \textit{Runtime}-driven over the baseline is reduced and stabilized close to 3.2$\times$ while {\proposal} continues to speedup the execution. 
Transferring all parts of the runtime to RTopt with the  \textit{Runtime}-driven approaches, leads to the serialization of the runtime.
Therefore, all scheduling operations (such as enqueue, dequeue of tasks, dependence analysis etc) that typically occur in parallel during runtime are executed sequentially on the RTopt.
Even if RTopt executes these operations faster than a general purpose core, serializing them potentially creates a bottleneck as we increase the number of cores.
{\proposal} does not transfer other runtime activities than the task creation, so it allows scheduling and dependence analysis operations to be performed in a distributed manner.

%We attribute this to the fact that serializing the scheduling operations becomes a bottleneck when increasing the number of cores.
%Scheduling operations (such as enqueue, dequeue of tasks, dependence analysis etc) generally occur in parallel during runtime, so serializing them for systems of up to 32 cores, is efficient.
%With an increased number of cores it is better to perform scheduling in a distributed manner, just as {\proposal} allows.
%
%Scheduling in general (enqueue, dequeue of tasks, dependence analysis etc) occurs in parallel during runtime.
%TaskGenX does not transfer the scheduling to the special hardware. So scheduling parts are executed on each core whenever they occur on the workers. The other approaches that do move the scheduling on the accelerator they serialize it because we assume that the accelerator is centralized. Is this clear? How could we put it clearly in the text?

%\begin{figure}[t]%
%	\centering
%	\includegraphics[width=0.6\textwidth]{figures/comparison.pdf}
%	\caption{Average improvement over baseline. X-axis shows the number of cores. For each case an RTopt core is used additionally to the number of cores.}
%	\label{fig:compare}
%	\vspace{-0.3cm}
%\end{figure}

\textit{Deps} driven approaches go through the same issue of the serialization of the dependency tracking and the scheduling that occurs at the dependence analysis stage.
The reason for the limited performance of \textit{Deps} compared to \textit{Runtime} is that \textit{Deps} does not accelerate any part of the task creation. 
Improvement over the baseline is still significant as performance with \textit{Deps} is improved by up to 1.5$\times$.

{\proposal} is the most efficient software-hardware co-design approach when it comes to highly parallel applications.
On average, it improves the baseline by up to 3.1$\times$ for homogeneous systems and up to 3.7$\times$ for heterogeneous systems.
Compared to other state of the art approaches, {\proposal} is more effective on a large number of cores showing higher performance by 54\% over \textit{Runtime} driven approaches and by 70\% over \textit{Deps} driven approaches.

%\begin{table*}[t]
%	\scriptsize
%	\begin{center}
%		\caption{Evaluated benchmarks and relevant characteristics}
%		\label{tab.taskgenx_cats}
%		%\resizebox{\textwidth}{!}{%
%			\begin{tabular}{|c|c|c|}
%				\hline
%				{\parbox{15mm}{\centering Application}} & 
%				{Min improvement} & 
%				{\parbox{20mm}{\centering Max Improvement}} \\
%				\hline	
%							
%				\multirow{2}{*}{\parbox{20mm}{\centering Cholesky 32K 256}} & 32K 256 & 357\,762   \\                                              
%				 & 32K 128 & 2829058 \\
%				\hline
%				\multirow{2}{*}{\parbox{18mm}{\centering Cholesky 32K 128}} & 16K 512 & 11\,442  \\
%				&  16K 128 & 707\,265 \\
%				\hline
%				QR 512 & native & 488\,202\\
%				\hline
%				QR 128 & & \\
%				\hline
%				Bodytrack & native & 329\,123 \\ 
%				%Heat diffusion & Heat &  &  &  &  &  & \\ 
%				\hline
%				Dedup & native & 20\,248  \\
%				\hline 
%				Ferret & native$\times$2 & 84\,002  \\
%				\hline
%		\end{tabular}%}
%	\end{center}
%\end{table*}
				
	
				

%%%%%%%%%%%%%%%%%%%%
%%%%%%%%%%%%%%%%%%%%
%\section{Related Work}
%\label{sec:related}
%The search for efficient task scheduling on multi-core systems has been intensively studied. Most scheduling heuristics target homogeneous multiprocessors, nevertheless there is an important number of studies in heterogeneous multiprocessors. In this section we give an overview of different categories of heterogeneous schedulers
% for heterogeneous systems
%, we explain some details about schedulers targeting specific systems using compute accelerators 
and explain details of previous works on criticality-aware schedulers.

%\textit{Schedulers for Heterogeneous Systems}
\textbf{Schedulers for Heterogeneous Systems: }
There are previous works on schedulers for heterogeneous systems that form four different types of schedulers: listing, clustering, guided-random, and duplication-based schedulers.

%do not consider the criticality of tasks~\cite{Hetero95, Dyn05, Gen07,Chemical, HEFT, Dup09}. These heuristics

Listing schedulers~\cite{List, DCPS, LDCP, HEFT, CrPathDup, Li2,Li5} have two scheduling stages. In the first stage, each task is given a priority based on the policy defined in each algorithm. In the second stage, tasks are assigned to processors depending on their priorities. Most criticality-aware schedulers fall in this category, and we discuss them in Section~\ref{sec.relwork_critical}. The scheduler proposed in this paper is also a list scheduler.

Clustering schedulers~\cite{Hypertool, DSC, DCPS, Hetero95} first separate tasks into clusters, where each cluster is to be executed on the same processor. During the clustering stage, the algorithm assumes an unlimited number of available processors in the system. If the number of clusters exceeds the number of available cores, the \textit{merging} stage joins multiple clusters so that they match the number of available processors. An example is the Levelized Min Time~\cite{Hetero95} clustering scheduler. This heuristic clusters tasks that can execute in parallel according to their \textit{level} (i.e. sibling nodes in a graph have the same level), and assigns priorities to the tasks in a cluster according to their cost, (i.e. tasks with the highest cost have the highest priority). The task-to processor assignment is done in decreasing order of priority.

Guided-random schedulers randomize their schedules by applying policies influenced by other sciences. Genetic algorithms~\cite{Gen07} group tasks into generations and schedule them according to a randomized genetic technique. Chemical reaction algorithms~\cite{Chemical, LiChemical} mimic molecular interactions to map tasks to processors. Some of these guided-random approaches are designed for heterogeneous systems~\cite{Gen07, Chemical}. The scheduler by Page et al.~\cite{Dyn05} enables dynamic scheduling of multiple-sized tasks for heterogeneous systems, but it lacks support of inter-task dependencies.

Duplication-based schedulers~\cite{Dup03, Dup11, Dup09} aim to eliminate communication costs between processors by scheduling tasks and their successors on the same processor. If a task has many successors, it is duplicated and executed in multiple cores prior to its successors to reduce communication costs.
%all successor tasks get the data from their predecessors with the lowest communication cost. 
This scheduling may introduce redundant task duplications tasks which may lead to bad schedules. The Heterogeneous Economical Duplication scheduler~\cite{Dup09} performs task duplication cautiously as it removes the redundant duplicates if they do not affect performance. 

These previous works schedule tasks statically and assume the prior knowledge of the task execution times on the different processor types in the heterogeneous system.

\if 0
\subsection{Schedulers for Compute Accelerators}

The schedulers in the previous section target the scheduling of generic TDGs on generic heterogeneous architectures. In this section we cover schedulers that target specific systems with compute accelerators. These works are more focused on the scheduling of tasks on the target platform based on the abstractions provided by the corresponding mixture of programming models for the general-purpose processors and the compute accelerators in the system.

Most heterogeneous systems with compute accelerators nowadays combine general-purpose CPUs and GPU compute accelerators. There is a set of programming models providing abstractions to ease the development of applications on these platforms. OmpSs~\cite{OmpSs_PPL11, OmpSs} offers this abstraction by allowing multiple implementations of a given task to be executed on different processing units~\cite{Judit}. The scheduler then assigns the execution of a task to the best resource according to its earliest finish time. Another case is StarPU~\cite{starpu}, a library that offers runtime heterogeneity support and provides priority schedulers for task-to-processor allocation. AHP~\cite{AHP} is another framework that generates software pipelines for heterogeneous systems and schedules tasks to their earliest executor, based on profiling information gathered prior to runtime.

None of these works, however, take into account the criticality of tasks regarding task dependencies, but they rather focus on the earliest execution time of individual tasks on the processor types in the specific system configuration.
\fi

%\subsection{Criticality-Aware Schedulers}
\label{sec.relwork_critical}
\textbf{Criticality-Aware Schedulers: }
Several previous works propose scheduling heuristics that focus on the critical path of a TDG to reduce total execution time~\cite{DCPS, LDCP, HEFT, CrPathDup, Moschakis2015}. To identify the tasks on the critical path, most of these works use the concept of \textit{upward rank} and \textit{downward rank}. The upward rank of a task is the maximum sum of computation and communication cost of the tasks in the dependency chains from that task to an exit node in the graph. The downward rank of a task is the maximum sum of computation and communication cost of the tasks in the dependency chain from an entry node to that task. Each task has an upward rank and downward rank for each processor type in the heterogeneous system, as the computation and communication costs differ across core types.

The Heterogeneous Earliest Finish Time (HEFT) algorithm~\cite{HEFT} maintains a list of tasks sorted in decreasing order of their upward rank. At each schedule step, HEFT assigns the task with the highest upward rank to the processor that finishes the execution of the task at the earliest possible time. Another work is the Longest Dynamic Critical Path (LDCP) algorithm~\cite{LDCP}. LDCP also statically schedules first the task with the highest upward rank on every schedule step. The difference between LDCP and HEFT is that LDCP updates the computation and communication costs on multiple processors of the scheduled task by the costs discovered in the processor to which it was assigned.

The Critical-Path-on-a-Processor (CPOP) algorithm~\cite{HEFT} also maintains a list of tasks sorted in decreasing order as in HEFT, but in this case it is ordered according to the addition of their \textit{upward rank} and \textit{downward rank}. The tasks with the highest \textit{upward rank + downward rank} belong to the critical path. On each step, these tasks are statically assigned to the processor that minimizes the critical-path execution time.

%The drawback of static listing algorithms is the static priority assignment can lead to wrong schedules at runtime~\cite{DCP}. 

The main weaknesses of these works are that (a) they assume prior knowledge of the computation and communication costs of each individual task on each processor type, (b) they operate statically on the whole TDG, so they do not apply to dynamically scheduled applications where only a part of the TDG is available at any given time, and (c) most of them use synthetic TDGs that are not necessarily representative of the dependencies in real workloads.

%These algorithms assume the prior knowledge of the execution time of each task in the task dependency graph. Moreover, they use static scheduling, namely the computation of the critical path and task-to-processor allocation are decided before the execution of the program. This can lead to wrong predictions of the critical path which, in practice, changes during runtime.





%%%%%%%%%%%%%%%%%%%
%%%%%%%%%%%%%%%%%%%
\section{Conclusions}
\label{sec.taskgenx.conclusions}

\section{Conclusions}
\label{sec.scheduling.conclusions}
We introduced the first critical-path-aware dynamic scheduler for heterogeneous systems as well as the first hybrid criticality-aware scheduler. Like CATS and contrary to previous works on criticality-aware scheduling that use synthetic TDGs and require prior knowledge of profiling information, our proposals work on real platforms with real applications and do not require off-line profiling.
%, they are implementable and work without the need of profiling.

We implemented and evaluated our scheduling proposals in the runtime system of the OmpSs programming model.
We showed that even if the accuracy of CPATH is higher in terms of task criticality identification, it does not always increase performance. 
Factors like the number of tasks and task cost variability play an important role on choosing the most appropriate scheduling policy and improve the performance of task-based applications.
The implementations shown in this Chapter will be included in the next stable release of the OmpSs programming model. 
Furthermore, the described policies are expected to be applicable to other task-based programming models with support for task dependencies. 
%The presented schedulers assume two core types.
%by manually choosing the cores that would act as fast, which limits the scheduling effect.
%To improve this, the schedulers can be modified to assign different levels of criticality to the tasks and let the cores, according to their type, execute the tasks with the corresponding criticality level.

%After determining the critical path, the scheduler could also consider the second, third etc. longest paths and insert their tasks in the corresponding ready queues. 
%The user has to specify which cores are considered as fast for the effective execution of the critical tasks. 
%We implemented and evaluated our criticality-aware task scheduler in the runtime system of the OmpSs programming model getting satisfactory results. The implementation shown in this paper will be included in the next stable release of OmpSs. Furthermore, there are no restrictions on applying our policy to other task-based programming models with support for task dependencies. 
    
%From our experiments on a real heterogeneous multi-core platform, we found a consistent performance improvement over the default breadth-first scheduling policy and a dynamic implementation of Heterogeneous Earliest Finish Time. The improvement of our proposal, which in most cases ranges from 10 to 20\% and reaches up to 30\%, is larger as we increase the number of cores. This gives a positive projection for CATS, as it is expected that the number of cores in multi-cores will increase throughout future generation designs.

%From our simulation experiments, we found out that the improvement of CATS increases over the baseline with larger differences of performance among fast and slow cores. We explored performance ratios between two and four times faster fast cores over slow cores, with improvements ranging from 30\% to 170\%. 

In conclusion, this chapter showed the potential of different heterogeneous schedulers to speed up dependency-intensive applications and take advantage of the asymmetric compute resources.


%For future work, we aim to extend the scheduling policy to be adaptive so it can dynamically adjust its flexibility and work stealing policy depending on the application characteristics and availability of resources at runtime.

%, and dynamically adapt its configuration to the one that best fits this combination.


